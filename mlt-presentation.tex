\documentclass[11pt]{beamer}

\usepackage{color,soul}

\mode<presentation>
{
  \usetheme{UIUC}
} 

\usepackage[utf8]{inputenc}
\usepackage{amsmath}
\usepackage{amsfonts}
\usepackage{amssymb}
\usepackage{algpseudocode} % uses algorithmicx package automatically
\usepackage{mathrsfs}
\usepackage{graphicx}
\DeclareMathOperator*{\argmin}{\mathbf{arg\,min}}
\DeclareMathOperator*{\argmax}{\mathbf{arg\,max}}
%\DeclareMathOperator*{\sup}{sup}
\DeclareMathOperator{\F}{\mathcal{F}} % Function Classes
\DeclareMathOperator{\FF}{\mathcal{F}} % prettier function classes
\DeclareMathOperator{\Hs}{\mathscr{H}} % Hilbert Spaces
\DeclareMathOperator{\R}{\mathbb{R}} % Reals
\DeclareMathOperator{\Rad}{\mathcal{R}} % Rademacher
\DeclareMathOperator{\E}{\mathbb{E}} % Expectation
\DeclareMathOperator{\Or}{\mathcal{O}} % Order Notation
\DeclareMathOperator{\Tr}{\textbf{Tr}} % Expectation
\DeclareMathOperator{\grad}{\nabla} % Gradient
\DeclareMathOperator{\LLH}{\mathcal{L}} % Log Likelihood etc.
\DeclareMathOperator{\Lag}{\mathcal{L}} % Lagrangian etc.
\DeclareMathOperator{\X}{\mathcal{X}} % input space X
\DeclareMathOperator{\Y}{\mathcal{Y}} % output space Y
\DeclareMathOperator{\bF}{\mathbf{F}} 
\DeclareMathOperator{\w}{\mathbf{w}} % weight vector
\DeclareMathOperator{\y}{\mathbf{y}} % output structure
\DeclareMathOperator{\x}{\mathbf{x}} % input structure
\DeclareMathOperator{\f}{\mathbf{f}} % function
\DeclareMathOperator{\K}{\mathcal{K}} % set of Kernels
\DeclareMathOperator{\vw}{\overrightarrow{w}} % set of Kernels
\DeclareMathOperator{\T}{\mathcal{T}} % set of Kernels

%\DeclareMathOperator{\implies}{\rightarrow} % implies

\newcommand{\opt}[1]{{#1}^{*}} % give f* for Optimal, Dual ...
\newcommand{\pred}[1]{\hat{#1}} % Prediction 
\newcommand{\dotprod}[2]{ \langle {#1} , {#2} \rangle } % <w,x> style
\renewcommand{\Pr}{\mathbb{P}} % Probability
\renewcommand{\vec}[1]{\mathbf{#1}} % vectors

\newcommand*{\Let}[2]{\State {#1} $\gets$ {#2}}

\usepackage{xcolor}

\newcommand{\highlight}[1]{\colorbox{yellow}{$\displaystyle #1$}}
\newcommand{\highlightmath}[1]{\colorbox{yellow}{\[\displaystyle #1\]}}

\author{Sham M. Kakade, Shai Shalev-Shwartz, Ambuj Tewari \newline Toyota Technological Institute-Chicago}


\title{On the duality of strong convexity and strong smoothness: Learning applications and matrix regularization}

\setbeamercovered{transparent} 
%\institute{UIUC} 
\date{Shyam Upadhyay (upadhya3), Adam Vollrath (vollrat2), Stephen Mayhew (mayhew2)} 
\begin{document}

{\nologo
\begin{frame}
\titlepage
\end{frame}
}

%\begin{frame}
%\tableofcontents
%\end{frame}


\begin{frame}{Introduction and definitions}
\begin{itemize}
\item Sub-differential (sub-gradient)
\item Convex Conjugate
\item Dual Norm
\item Strong Convexity
\item Smoothness
\end{itemize}
\end{frame}

\begin{frame}{Strong Convexity}
A function is said to be $\beta$ strongly convex if,
\begin{align*}
\f(\x+\y) \ge \underbrace{\f(\x) + \dotprod{\grad \f(\x)}{\y}}_{\text{the value of the tangent}} +\frac{\beta}{2} \|\y\|^2
\end{align*}

\end{frame}

\begin{frame}{Smoothness}
A function is said to be $\beta$ smooth if,
\begin{align*}
\f(\x+\y) \le \f(\x) + \dotprod{\grad \f(\x)}{\y} +\frac{\beta}{2} \|\y\|^2
\end{align*}
\end{frame}

\begin{frame}{Strong/Smoothness Duality}
\begin{theorem}
Assume $\f$ is closed and convex function. Then
\begin{align*}
\f \text{ is } \beta\text{-strongly convex} \iff \opt{\f} \text{ is } \frac{1}{\beta}\text{-smooth}
\end{align*}
\end{theorem}
\end{frame}

\begin{frame}{A Useful Lemma}
\end{frame}

\begin{frame}{Generalized Regret Bound}
\end{frame}


\begin{frame}{Rademacher Bounds}
\end{frame}


%% Adam Vollrath's part of the presentation.


\begin{frame}{Group Lasso}
Let  $X = (X_1, \dots, X_d)$ be an $k \times d$ real matrix with columns
$X_i \in \R^k$. The group Lasso, $\|X\|_{2,1}$, is the $L_1$-norm of the $L_2$-norms of
the columns:
%\[|(\|X_1\|_2, \dots, \|X_n\|_2)\|_1;\]
\[\sum_{i = 0}^d \|X_i\|_2.\]
%\vspace{0.2in}
This is useful in learning matrices, $W \in \R^{k \times d}$, where the objective function is
\[ \min_W \frac{1}{n} \sum_{i=1}^n \|\y_i - W\x_i\|_2^2 + \lambda \|W\|_{2,1}.\]
\end{frame}


\begin{frame}
\begin{theorem}
Let the distribution of $\x \in \R^d$ and $\y \in \R^k$ be such that
$\|\x\|_\infty \le X_\infty$ and $\|\y\|_2 \le Y_2$ almost surely.
Then for the function class
\[\F = \{(\x, \y) \mapsto \|\y - W\x\|_2^2 : \|W\|_{2,1} \le \bar{W}_{2,1}\}\]
for some $W \in \R^{k \times d}$ and $\bar{W}_{2,1} > 0$, we have
\[\Rad_n(\F) \le \frac{(Y_2 + e\bar{W}_{2,1} X_\infty \sqrt{\log d})^2}{\sqrt{n}}.\]
\end{theorem}
%If only $q$ shared weights matter, 
\end{frame}


\begin{frame}
\frametitle{Proof}
\begin{align*}
\|\y - W\x\|_2^2 &= \y^T \y - 2 \y^T W \x + \x^T W^T W \x \\
   &= \y^T \y - 2 Tr(\y^T W \x) + Tr(\x^T W^T W \x) \\
   &= \y^T \y - 2 Tr(\x \y^T W) + Tr(W^T W \x \x^T) \\
   &= \y^T \y - 2\langle \y \x^T, W\rangle + \langle W^TW, \x \x^T\rangle
\end{align*}
Consider the function classes
\begin{align*}
\F_1 &= \{(\x, \y) \mapsto 2\langle W, \y\x^T\rangle : \|W\|_{2,1} \le \bar{W}_{2,1}\} \\
\F_2 &= \{(\x, \y) \mapsto 2\langle W^T W, \y\x^T\rangle : \|W\|_{2,1} \le \bar{W}_{2,1}\}
\end{align*}
Can show $\Rad_n(\F_1) \le \Rad_n(F_1) + \Rad_n(F)$.
\end{frame}


\begin{frame}
\frametitle{Proof Continued ($\Rad_n(\F_1)$ Upper Bound)}
If $r \in (1,2]$ and $1/r + 1/s = 1$ then $\frac{1}{2} \| \cdot \|_{2,s}^2$ is $s$-smooth and so
its conjugate $\frac{1}{2}\| \cdot \|_{2,r}^2$ is $1/s$-strongly convex. So,
setting $f(W) = \frac{1}{2}\|W\|_{2,r}^2$ and $\| \cdot \| = \| \cdot \|_{2,r}$ in 
the previous theorem gives
\[\Rad_n(\F_1) \le 2\|y\x^T\|_{2,s}\bar{W}_{2,1}\sqrt{\frac{s}{n}} \le 
   2d^{1/s}Y_2X_\infty \bar{W}_{2,1}\sqrt{\frac{s}{n}}.\]
Set $s = \log d$ to get
\[\Rad_n(\F_1) \le 2eY_2X_\infty \bar{W}_{2,1}\sqrt{\frac{\log d}{n}}.\]
\end{frame}


\begin{frame}
\frametitle{Proof Continued ($\Rad_n(\F_2)$ Upper Bound)}
For $\F_2$,
\begin{align*}
\|W^TW\|_{1,1,} &= \sum_{i,j} |\langle W_{\cdot,i},   W_{\cdot,j}\rangle| \\
  &\le \sum_{i,j} \|W_{\cdot,i}\|_2 \cdot \|W_{\cdot,j}\|_2 \\
  &= \sum_i \|W_{\cdot,i}\|_2 \cdot \sum_j \|W_{\cdot,j}\|_2 \\
  &= \|W\|_{2,1}^2
\end{align*}
Using Kakade et al. [2008] and $\|\x\x^T\|_{\infty,\infty}$ gives
\[\Rad_n(\F_2) \le X_\infty^2 \bar{W}_{2,1}^2 \sqrt{\frac{2\log d}{n}}.\]
\end{frame}




\begin{frame}{Kernel Learning}

  We follow the definition of learning kernels given in Lanckriet et al. \cite{lanckriet2004learning}.

  Choose a set of kernels $\{K_1, K_2, \cdots K_k\}$, and consider convex combinations:

  \[ \mathcal{K}^+_c = \left\{ \sum^k_{j=1} \mu_j K_j ~:~ \mu_j \geq 0, \sum^k_{j=1} \mu_j = 1 \right\}  \]

  We want to learn the $\mu_j$ parameters. 

\end{frame}


\begin{frame}{Class to consider}

  \begin{multline}
    \mathcal{F}_{\mathcal{K}^+_c} = \{ \mathbf{x} \mapsto \sum_{i=1}^n \alpha_i K(\mathbf{x}_i, \cdot) : \\
    K = \sum_{j=1}^k \mu_j K_j, \\
    \mu_j \geq 0,\\
    \sum_{j=1}^k \mu_j = 1, \\
    \boldsymbol\alpha^T K(\mathcal{T}) \boldsymbol\alpha \leq 1 / \gamma^2 \\
    \} 
  \end{multline}
  
\end{frame}

\begin{frame}{Rademacher complexity}


  \begin{theorem}[21, Kernel Learning]
    Consider the class $\mathcal{F}_{\mathcal{K}^+_c}$ defined above. Let $K_j(\x,\x) \leq B$ for $1 \leq j \leq k$ and $\x \in \X$. Then
    \[ \Rad_{\mathcal{T}}(\mathcal{F}_{\mathcal{K}^+_c}) \leq e \sqrt{\frac{B \log k}{\gamma^2 n}} \]
  \end{theorem}
  
  Bound on $k$ is logarithmic: we can use a large number of base kernels!
  
\end{frame}


\begin{frame}{Proof}

  \begin{theorem}[9 Generalization]
    Lef $f$ be a $\beta$-strongly convex function w.r.t. a norm $\|\cdot \|$ on $S$ such that
    $f^*(\mathbf{0}) = 0$. Let $\X = \{ x : \|x\|_* \leq X\}$ and $\mathcal{W} = \{ w : f(w) \leq f_{max} \}$.
    Consider the class of linear functions,

    \[ \mathcal{F} = \{ x \mapsto \langle w,x \rangle : w \in \mathcal{W} \}  \]

    Then, for any dataset $\mathcal{T} \in \X^n$, we have

    \[ \Rad_{\mathcal{T}}(\mathcal{F}) \leq X \sqrt{ \frac{2f_{max}}{\beta n}  } \]
  \end{theorem}
      

  
\end{frame}

\begin{frame}{Filling in the blanks...}

  First condition:
  
  \[ \X = \{ x ~:~ \|x\|_* \leq X\} \]

  Satisfied by:

  \[ \X = \{ x ~:~ \|\phi(x)\|_{2,s} \leq k^{1/s} \sqrt{B}\} \]

  
\end{frame}

\begin{frame}{Filling in the blanks...}

  Second condition:

  \[ \mathcal{W} = \{ w : f(w) \leq f_{max} \}  \]

  Satisfied by:
  
  \[ \mathcal{W} = \{ \vw ~:~ \| \vw\|^2_{2,r} \leq 1/\gamma^2 \}  \]
  
\end{frame}


\begin{frame}{Filling in the blanks...}

  And this class of functions?
  
  \[ \mathcal{F} = \{ x \mapsto \langle w,x \rangle : w \in \mathcal{W} \}  \]

  We can say: $ \FF_{\mathcal{K}^+_c} \subseteq \FF_r $ for

  \[ \FF_r := \{ \x \mapsto \langle \vw, \phi(\x) \rangle ~:~ \vw \in \Hs, ~ \| \vw \|_{2,r} \leq 1/\gamma \} \]
  
  
\end{frame}



\begin{frame}{Where are we?}

  \begin{theorem}[9 Generalization]
    Lef \highlight{f} be a $\beta$-strongly convex function w.r.t. a norm $\|\cdot \|$ on $S$ such that
    $f^*(\mathbf{0}) = 0$. \highlight{\X = \{ x : \|x\|_* \leq X\}} and \highlight{\mathcal{W} = \{ w : f(w) \leq f_{max} \}}.
    Consider the class of linear functions,

    \[ \mathcal{F} = \{ x \mapsto \langle w,x \rangle : w \in \mathcal{W} \}\]

    Then, for any dataset $\mathcal{T} \in \X^n$, we have

    \[ \Rad_{\mathcal{T}}(\mathcal{F}) \leq X \sqrt{ \frac{2f_{max}}{\beta n}  } \]
  \end{theorem}
      
\end{frame}


\begin{frame}{Last condition...}

  Finally, we need to show that $f$ is $\beta$-strongly convex. By
  
  \begin{corollary}[19]
  Let $q,s \geq 2$. The function $\frac{1}{2}\|\cdot \|^2_{q,s}$ is $(q+s-2)$-smooth w.r.t. $\|\cdot \|_{q,s}$ on $\R^{m \times n}$.
  \end{corollary}
  
  we know that $\frac{1}{2} \| \cdot \|^2_{2,s}$ is $s$-smooth. Thus, by Theorem 6, its conjugate, $\| \cdot \|_{2,r}^2$, is $1/s$-strongly convex. 

  \vspace*{0.5cm}
  
  That is, $f = \| \vw\|^2_{2,r}$, and is $1/s$-strongly convex.

  
\end{frame}


\begin{frame}{Plug and Chug}

  Substitute $X = k^{1/s}\sqrt{B}, f_{max} = \frac{1}{2}\gamma^2, \beta = 1/s$ into 

    \[ \Rad_{\mathcal{T}}(\mathcal{F}) \leq X \sqrt{ \frac{2f_{max}}{\beta n}  } \]

    And we have:

    \[ \Rad_{\mathcal{T}}(\mathcal{F}_{\mathcal{K}^+_c}) \leq e \sqrt{\frac{B \log k}{\gamma^2 n}} \]
    
\end{frame}


\begin{frame}[allowframebreaks]{References}
  \def\newblock{}
  \bibliographystyle{plain}
  \bibliography{mlt-presentation.bib}
\end{frame}

\end{document}
